\chapter{History Background}
\newpage
\section{OTA for Win32}
\subsection{Before Delphi 4 (obsolete)} 
Tool vendors can never meet every needs of their clients. As a result, they
provide the clients some way to extend the products. Since the very first version of
Delphi, OTA has been available so that developers could extend the IDE.

The old OTA for Delphi was very complex according to Erik Berry because at the 
beginning, the Delphi language did not support real interfaces (C++ does not
support interfaces either). That kind of OTA is not available in Galileo IDE, so
you cannot use it now, and details of it is not documented here.
\begin{sffamily} 

\begin{quote} 
The older OTA is depreciated and should no longer be used except 
to maintain compatibility with older IDE versions. Support for the old OTA will 
be dropped in a future version of Delphi, and existing bugs in it are not being 
fixed. The old Open Tools API consists of the following units: ExptIntf, 
FileIntf, IStreams, ToolIntf, VcsIntf, VirtIntf.
\end{quote}
--- taken from ''Erik's Open Tools API FAQ and Resources'' on 
\underline{gexperts.org}.
\end{sffamily}

\subsection{From Delphi 4 On}
\begin{sffamily}
\begin{quote} 
This FAQ only covers the ''new'' Open Tools API (OTA), which 
consists of the ToolsAPI.pas unit in Delphi 4 or greater and C++Builder 4 or 
greater. Starting with Delphi 8, there are a few more files such as 
PaletteAPI.pas and PropInspAPI.pas.
\end{quote}
--- taken from ''Erik's Open Tools API FAQ and Resources'' on 
\underline{gexperts.org}.
\end{sffamily}

When real interfaces support was added to Object Pascal language, Borland R\&D
started to provide Open Tools Architecture (for Win32). This ''OTA'' is too
powerful to miss. However, it is hard to master. Once you know a lot about it,
you can do some amazing things.

CodeRush was one of the famous Delphi plug in. Its author, Mark Miller, used OTA
very well and designed a flexible architecture for CodeRush. The users of
CodeRush could even write plug in to extend CodeRush. This is why Mark invented
a series of APIs himself, named RushAPI (for Win32). Before Delphi 8 was
shipped, Mike's company was Merged with Developer Express, Inc. Since then no
breaking version of CodeRush had come except the versions for Visual Studio
.NET. It is said that CodeRush used too much subclassing so when Galileo came it
was rather hard to port to the new IDE. It is a pity that we cannot use CodeRush
or RushAPI to extend Delphi 8/2005/2006.

Later other commercial OTA products such as Castalia has implemented similar 
architectures. However, they are not open to other developers compared to CodeRush.

Open source experts like GExperts and CnPack IDE Wizards are easy to extend, but
still not as flexible as CodeRush. The main problem is that you can not just
write a plug in for them which is loaded by them at run time. New features
should be added in the project and compile with others.

In all, I found no easy way to build an SDK for OpenTools development in Win32
world.

\section{OTA for .NET}
First introduced in C\#Builder 1.0, Borland Developer Studio IDE offers OTA for 
.NET, or Managed OTA, or Galileo OTA. This helps C\# users to extend the IDE.
And there was even a contest for writing such kind of experts.

When designing ConvertAddIn and FavouritesMenu, Marc Rohloff had implemented a 
special architecture for his .NET OTA tools. It is open source but I don't 
notice its existence until recently. I am already on SBT's way so far and I 
cannot turn back. Mark's architecture reduces dependency on OTA APIs, but I 
don't know whether I can build LeXDK upon his architecture. Marc won the second
prize in the contest for his Favourites Menu.

David Hervieux created Sharp Builder Tools for C\#Builder since its version 1. 
The latest version of SBT is 3.1. This version is too old to support Delphi 
2005/2006 (in fact after only a few modifications it can). Although David won 
the first prize in the OTA contest held by Borland, he failed to continue this 
project for ''too many problems with Borland''.\footnote{See License.pdf for 
details.}

After digging David's codebase I found it was easy to extend SBT, too. Code
Beautifier Collection (originally named JCFExpert) was built then. At that
moment, I had to mix my code with David's. That is why I decided to bring the
codebase further to make an SDK for building OpenTools immediately. 

David has set up a lot of patterns, while .NET Framework provides powerful
reflection interfaces, it is easy to make a plug-in architecture come true at
last. It can really reduce the complexity of extending the Galileo IDE. So I
build later versions of Code Beautifier Collection on this SDK. This SDK is
officially named as LeXtudio OpenTools SDK, LeXDK for short.

Many people may fear to use this SDK because they think it reduces their freedom. 
Actually, LeXDK does not prevent you from developing directly with Galileo OTA.
You can directly use OTA when you develop a feature. Many existing features serve as
good examples. Using LeXDK, you can get rid of duplicate code and enjoy a lot of
generic services, such as logging and preference management.

There are still some limitations, but they are temporary because I have not got
time to implement certain features.

Remember this SDK is improving in every new versions.

This document focuses on telling you how to start a plug in project for LeXDK,
or Code Beautifier Collection. In this way you can easily extend the function of
Galileo IDE without starting from the scratch.
