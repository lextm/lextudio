
\chapter{Plus Registration File Format}

\newpage
\section{Specifications Version I (out-of-date)}
From 5.1.1, this version of .plus file is not used any more.

You should refer to Specifications Version II.

\subsection{Why Version I is Simple}
Actually, I knew little about file manipulations before I designed this
architecture. So I have to first do some experiments in order to make myself
familiar with this staff. A simple version I of this file format can already
provide basic configuration ability for Plus. As a result, I take it as the
specifications.

Later version should be more complex, but still easy to configure.

\subsection{File Extension}
.plus is the file extension.

\subsection{How to Generate}
The file should be a text file.

The file name should be the same with the corresponding Plus assembly file.

Each lines is for a feature.

One line should contains two sections separated by a semicolon.

The first section is the name of the feature class.

The second is a flag. Set it to true for enabled and false for disabled.

\begin{quotation}
For example, this is how I write for Code Beautifiers Plus.

The file name should be Lextm.CodeBeautifiers.Plus.plus.

In the file there is only one line:

Lextm.CodeBeautifierCollection.Features.Code\-Beauti\-fiers;\-true
\end{quotation}

\section{Specifications Version II}

\subsection{What's missing in Version I}
\index{Plus registration}
Basic functions are done in version I. However, it is not easy to configure a
feature freely even with a GUI tool (finally I wrote a Plus Manager to do
this). The file format lacks a few flexibility in fact. So this version II
should go further.

It is sometimes comfortable to enable one feature in BDS 1 and disabled it in
BDS 2. So Version II takes this need into account. For example, FileWizards
feature in WiseEditor Plus cannot run under BDS 1.0 because of some unsupported
interfaces. So it is better that the enabled flag can be set for every BDS
version.

Also, there are a few calls for personality level enabled/disabled control.
Such a further requirement may be included in this version or in a later
version III.

In order to make this format more powerful, simple text file may not be suited.
So it should be in INI format or XML format. XML file is easy for serialization
and deserialization, so .plus2 file is an XML file.

\subsection{File Extension}
.plus2 is the file extension.

\subsection{How to Generate}
\begin{quote}
\begin{flushleft}
\begin{sffamily}
$<$Plus2$>$\\
\tab $<$ModuleName$>$G:$\backslash$lextm$\backslash$SharpDevelop
Projects$\backslash$XmlReader$\backslash$bin$\backslash$Debug$\backslash$Lextm.CodeBeautifiers.Plus$<$/ModuleName$>$\\
\tab $<$Name$>$CodeBeautifiers$<$/Name$>$\\
\tab $<$Features$>$\\
\tab \tab $<$Feature2$>$\\
\tab \tab \tab $<$EnabledRecords$>$\\
\tab \tab \tab \tab $<$EnabledRecord$>$\\
\tab \tab \tab \tab \tab $<$Version$>$1$<$/Version$>$\\
\tab \tab \tab \tab \tab $<$Enabled$>$true$<$/Enabled$>$\\
\tab \tab \tab \tab $<$/EnabledRecord$>$\\
\tab \tab \tab \tab $<$EnabledRecord$>$\\
\tab \tab \tab \tab \tab $<$Version$>$2$<$/Version$>$\\
\tab \tab \tab \tab \tab $<$Enabled$>$true$<$/Enabled$>$\\
\tab \tab \tab \tab $<$/EnabledRecord$>$\\
\tab \tab \tab \tab $<$EnabledRecord$>$\\
\tab \tab \tab \tab \tab $<$Version$>$3$<$/Version$>$\\
\tab \tab \tab \tab \tab $<$Enabled$>$true$<$/Enabled$>$\\
\tab \tab \tab \tab $<$/EnabledRecord$>$\\
\tab \tab \tab \tab $<$EnabledRecord$>$\\
\tab \tab \tab \tab \tab $<$Version$>$4$<$/Version$>$\\
\tab \tab \tab \tab \tab $<$Enabled$>$true$<$/Enabled$>$\\
\tab \tab \tab \tab $<$/EnabledRecord$>$\\
\tab \tab \tab $<$/EnabledRecords$>$\\
\tab \tab \tab $<$Name$>$Lextm.CodeBeautifierCollection.Features.
Code\-Beaut\-ifi\-ers$<$/Name$>$\\
\tab \tab \tab $<$Description$>$This feature invokes several tools to format
current file or projects.\\
Ctrl + W is its shortcut.$<$/Description$>$\\
\tab \tab $<$/Feature2$>$\\
\tab $<$/Features$>$\\
$<$/Plus2$>$
\end{sffamily}
\end{flushleft}
\end{quote}

The above is an example .plus2 file, named Lextm.CodeBeautifiers.Pl\-us.pl\-us2.

The file name should be the same with the corresponding plus assembly file,
only the extension changes to .plus2.

When loaded, this file is deserialized to an object of class Plus2.

$<$Plus2$>$ tag indicates that it is a Plus2 object.

$<$ModuleName$>$ tag refers to its plus assembly name without extension.

$<$Name$>$ tag stores the plus' name.

$<$Features$>$ tag contains many instance of class Feature2.

$<$Feature2$>$ tag indicates that it is a Feature2 object.

$<$EnabledRecords$>$ tag stores a few EnabledRecord objects.

$<$EnabledRecord$>$ tag indicates that it is a EnabledRecord object.

$<$Version$>$ tag refers to IDE version, 1 to 5.

$<$Enabled$>$ tag contains the flag, true or false.

$<$Name$>$ and $<$Description$>$ tags stores feature information.